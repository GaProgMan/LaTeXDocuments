%%% LaTeX Template: Curriculum Vitae
%%%
%%% Source: http://www.howtotex.com/
%%% Feel free to distribute this template, but please keep the referal to HowToTeX.com.
%%% Date: July 2011

%%% Edited by Jamie Taylor (http://gaprogman.com)
%%% Added a definition for a "hobbies" section
%%% Added pdfinfo call to add metadata to outputted PDF
%%% Date September 2012

%%% ------------------------------------------------------------
%%% BEGIN PREAMBLE
%%% ------------------------------------------------------------
\documentclass[paper=a4,fontsize=10pt]{scrartcl}		% KOMA-article class
							
\usepackage{amsmath,amsfonts,amsthm}				% Math packages
\usepackage[pdftex]{graphicx}					% Enable pdflatex
\usepackage[svgnames]{xcolor}					% Colors by their 'svgnames'
\usepackage[margin=0.5in]{geometry}
	\textheight=700px					% Saving trees ;-) 
\usepackage{url}						% Clickable URL's
\usepackage{wrapfig}						% Wrap text along figures

\frenchspacing							% Better looking spacings after periods
\pagestyle{empty}						% No pagenumbers/headers/footers
%\usepackage{bbding}						% Symbols

%%% pdfstartview=Fitforces the outputted pdf to open in a "fit
%%% to windows" zoom level.
%%% hidelinks tells pdflatex to NOT draw borders around any and
%%% all links in the compiled document
\usepackage[pdfstartview=Fit, hidelinks]{hyperref}

\hypersetup{pdfinfo={
Title={Resume.pdf},
Author={Author Name},
Creator={LaTeX},
Producer={TeX Maker 3.5},
Subject={Curriculum Vitae}
}}
  
%%% Custom sectioning (sectsty package)
%%% ------------------------------------------------------------
\usepackage{sectsty}						% Custom sectioning (see below)

\sectionfont{%							% Change font of \section command
	\usefont{OT1}{phv}{b}{n}%				% bch-b-n: CharterBT-Bold font
	\sectionrule{0pt}{0pt}{-5pt}{3pt}
	}

%%% Macros
%%% ------------------------------------------------------------
\newlength{\spacebox}
\settowidth{\spacebox}{8888888888}			  	% Box to align text
\newcommand{\sepspace}{\vspace*{1em}}			  	% Vertical space macro

\newcommand{\MyName}[1]{
		\Large \usefont{OT1}{phv}{b}{n} \hfill #	% Name
		\par \normalsize \normalfont}
		
\newcommand{\MySlogan}[1]{
%		\large
		\usefont{OT1}{phv}{m}{n}\hfill \textit{#1}	% Slogan (optional)
		\par \normalsize \normalfont}

\newcommand{\NewPart}[1]{\section*{\uppercase{#1}}}

\newcommand{\PersonalEntry}[2]{
		\noindent\hangindent=2em\hangafter=0 		% Indentation
		\parbox{\spacebox}{				% Box to align text
		\textit{#1}}					% Entry name (birth, address, etc.)
		\hspace{1.5em} #2 \par}				% Entry value

\newcommand{\SkillsEntry}[2]{					% Same as \PersonalEntry
		\noindent\hangindent=2em\hangafter=0 		% Indentation
		\parbox{\spacebox}{				% Box to align text
		{#1}}						% Entry name (birth, address, etc.)
		\hspace{1.5em} #2 \par}				% Entry value	
		
\newcommand{\EducationEntry}[5]{
		\noindent \textbf{#1} \hfill			% Study
		#2 \par						% Duration
		\noindent #3 					% school
		\hfill #4 \par					% School
		\noindent\hangindent=2em\hangafter=0 \small #5 	% Description
		\normalsize \par}
		
\newcommand{\EducationEntryNoDetail}[4]{
		\noindent \textbf{#1} \hfill			% Study
		#2 \par						% Duration
		\noindent #3 					% School Name
		\hfill #4 \par					% School NAme
		\noindent\hangindent=2em\hangafter=0 \small
		\normalsize \par}


\newcommand{\ExperienceEntry}[4]{				% Same as \EducationEntry
		\noindent \textbf{#1} \hfill 			% Jobname
		#2 \par						% Duration
		\noindent #3 \par				% Company
		\noindent\hangindent=2em\hangafter=0 \small #4 	% Description
		\normalsize \par}
		
\newcommand{\ExperienceEntryNoCompanyName}[3]{			% Same as \EducationEntry
		\noindent \textbf{#1} \hfill 			% Jobname
		#2 \par						% Duration
		\noindent\hangindent=2em\hangafter=0 \small #3 	% Description
		\normalsize \par}
		
\newcommand{\GitHubLink}[2]{					% Used to display a textual description of a GitHub link
		\noindent #1 \hfill 				% Description or name
		\textit{\href{#2}{#2}} \par			% Link
		\normalsize \par}		
		
\newcommand{\OnlineEntry}[3]{					% Similar to Hobby entry
		\noindent \textbf{#1} \hfill 			% Service/site name
		\noindent \textit{\href{#2}{#2}} \par		% URL
		\noindent \hangindent=2em \hangafter=0 \small #3%Description
		\normalsize \par}

\newcommand{\ProjectDetails}[3]{ 				% Used for project information
		\textbf{#1} \hfill 				% Name of Project
		#2 \par						% Languages, Tool, etc. used
		\noindent\hangindent=2em\hangafter=0 \small #3 	% Description
		\normalsize \par}

%%% ------------------------------------------------------------
%%% HobbyEntry COMMAND
%%% ------------------------------------------------------------
%%% Custom command added by Jamie Taylor
%%% Used to add detail to the hobby section so that it takes 
%%% a similar format to the other sections - in an effort to
%%% maintain the style of the overall document.
%%% Useage:		\HobbyEntry{type of hobby}{description of hobby}
%%% Example:	\HobbyEntry{Cooking}{Cake is nice; I like baking}

%\newcommand{\HobbyEntry}[2]{					% hobbies Section
%		\noindent \textbf{#1} \hfill \par		% hobby classification
%		\noindent\hangindent=2em\hangafter=0 \small #2 	% Description
%		\normalsize \par}


%%% ------------------------------------------------------------
%%% BEGIN DOCUMENT
%%% ------------------------------------------------------------
\begin{document}

\MyName{Name}
\MySlogan{Contact Number}
\MySlogan{\href{mailto:email@email.com}{email@email.com}}
\MySlogan{Full Address Here}

%%% Work experience
%%% ------------------------------------------------------------
\NewPart{Experience}
{
	\ExperienceEntry{Job Title}{Date from - Date to}{Company Name}{

	\begin{itemize}
		\item{Very short}
		\item{List of projects}
		\item{Undertaken at}
		\item{Company}
	\end{itemize}
	\sepspace

		\ProjectDetails{Short Description - Name of Project}{\textsc{Tools}, \textsc{And}, \textsc{Languages}, \textsc{Used}}
		{
			Description of the project, how you fit into it, which tools were used and where and how the project changed as you worked on it.
			\\ Maximum of two very short paragraphs of text.
		}
		\sepspace
		\ProjectDetails{Short Description - Name of Project}{\textsc{Tools}, \textsc{And}, \textsc{Languages}, \textsc{Used}}
		{
			Description of the project, how you fit into it, which tools were used and where and how the project changed as you worked on it.
			\\ Maximum of two very short paragraphs of text.
		}
		\sepspace
		\ProjectDetails{Short Description - Name of Project}{\textsc{Tools}, \textsc{And}, \textsc{Languages}, \textsc{Used}}
		{
			Description of the project, how you fit into it, which tools were used and where and how the project changed as you worked on it.
			\\ Maximum of two very short paragraphs of text.
		}
		\sepspace
		\ProjectDetails{Short Description - Name of Project}{\textsc{Tools}, \textsc{And}, \textsc{Languages}, \textsc{Used}}
		{
			Description of the project, how you fit into it, which tools were used and where and how the project changed as you worked on it.
			\\ Maximum of two very short paragraphs of text.
		}
		\sepspace
		\ProjectDetails{Short Description - Name of Project}{\textsc{Tools}, \textsc{And}, \textsc{Languages}, \textsc{Used}}
		{
			Description of the project, how you fit into it, which tools were used and where and how the project changed as you worked on it.
			\\ Maximum of two very short paragraphs of text.
		}
		\sepspace
		
		\par Paragraph on development team techniques used for projects (Agile, etc.). If none were used, describe the typical development life cycle.

%%% Skills
%%% ------------------------------------------------------------
\NewPart{Key Skills}
{
	\SkillsEntry{Programming Languages}{\textsc{List of}, \textsc{languages you}/\textsc{are familiar}, \textsc{woth. Ranging}, \LaTeX, \textsc{from most familiar}, \textsc{and pften used}, \textsc{to least}}
	\sepspace
	\SkillsEntry{Operating Systems}{Same as Programming Languages}
}

%%% Education
%%% ------------------------------------------------------------
\NewPart{Education}
{ 
	\EducationEntry{Course Title}{Date from - Date to}{Institution of study}{}
	{
		% Optional description of course.
	}
	
	% Use as many as are relevant
}

\NewPart{Online Presence}
{
	\OnlineEntry{Website}{link here}
	{
		Brief description of what can be found at the link
	}
	\sepspace

	\OnlineEntry{Github}{link here}
	{
		It is often useful to give interviewers a chance to check out some open source code that you may have written. Give a brief description of some of your best code here (this may be asked about at interview)
	}
	\sepspace

	\OnlineEntry{Stack Exchange}{link}
	{
		Being part of a community, extending your skill set and answering questions on techniques and languages can be very helpful to an interviewer in getting to know you on a technical level. Intice them to check out your community presence here.
	}
	\sepspace

	\OnlineEntry{LinkedIn}{link}
	{
		If you have it, link it. This can show your business connections and your previous employment. Which will make attaining references easier.
	}
}


%%% GitHub
%%% ------------------------------------------------------------
\NewPart{Selection of GitHub Projects}
{
	% This is a chance to expand on thescant details provided in the section above
	
	\OnlineEntry{Name of project}{Link here}
	{
		Which languages did you use? How did you go about producing this code? Was it collaborative? Was it entirely in your own time? Did you have a specific goal in mind (was it produced for a friend/family member or was it just an experiment?)?
	}
	\sepspace
	\OnlineEntry{Name of project}{Link here}
	{
		Which languages did you use? How did you go about producing this code? Was it collaborative? Was it entirely in your own time? Did you have a specific goal in mind (was it produced for a friend/family member or was it just an experiment?)?
	}
	\sepspace
	\OnlineEntry{Name of project}{Link here}
	{
		Which languages did you use? How did you go about producing this code? Was it collaborative? Was it entirely in your own time? Did you have a specific goal in mind (was it produced for a friend/family member or was it just an experiment?)?
	}
	\sepspace
	\OnlineEntry{Name of project}{Link here}
	{
		Which languages did you use? How did you go about producing this code? Was it collaborative? Was it entirely in your own time? Did you have a specific goal in mind (was it produced for a friend/family member or was it just an experiment?)?
	}
	\sepspace
	\OnlineEntry{Name of project}{Link here}
	{
		Which languages did you use? How did you go about producing this code? Was it collaborative? Was it entirely in your own time? Did you have a specific goal in mind (was it produced for a friend/family member or was it just an experiment?)?
	}
	\sepspace
}

%%% ------------------------------------------------------------
%%% END DOCUMENT
%%% ------------------------------------------------------------
\end{document}
